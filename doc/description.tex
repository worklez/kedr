\documentclass[a4paper,12pt,oneside]{article} 

\usepackage[utf8]{inputenc}
\usepackage[russian]{babel} 
\usepackage{cmap} % only with pdftex
\usepackage{indentfirst}

\begin{document}

\section*{KEDR --- Keep 'em Diabolically Right}
\subsection*{Формулировка задачи}

Одной из актуальных проблем как любого меломана, так и рядового
пользователя с распространением хранения музыки на жестких
дисках стал процесс правильного подписывания метаинформации (<<тегов>>) к
музыкальным трекам. Большинство популярных программ подразумевают ручное
заполнение пользователем <<тегов>> по одному файлу, что является очень
трудоемким процессом и часто ведет к ошибкам.

Идея заключается в создании программы для заполнения <<тегов>> аудиофайлов
нового поколения. Использование альбом-ориентированной концепции
позволит заполнять теги сразу для всего альбома при помощи
<<отпечатков>> песен и указаний пользователя. В качестве базы песен
предполагается использоваться онлайн-базу musicbrainz.org, поэтому
программе будет требоваться подключение к интернету.

Программа должна поддерживать основные платформы, такие как GNU/Linux, Mac OS X,
Windows, и основные форматы аудиофайлов, такие как FLAC, Ogg Vorbis,
MP3.

\end{document}
