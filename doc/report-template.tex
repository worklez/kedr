\documentclass[pdftex,12pt,a4paper]{report}


\usepackage[utf8]{inputenc}
\usepackage[russian]{babel} 
\usepackage[pdftex]{graphicx}
\usepackage{cmap} % only with pdftex
\usepackage{indentfirst}
\usepackage{color}
\usepackage[left=3cm,top=2cm,bottom=2cm,right=1cm,nohead,nofoot]{geometry}
\usepackage[nofancy]{svninfo}

\providecommand{\comment}[1]{\textcolor{blue}{#1}}
\providecommand{\replace}[1]{\textcolor{green}{#1}}

\renewcommand{\thesection}{\arabic{section}}

\setcounter{secnumdepth}{5}
\setcounter{tocdepth}{5}

\usepackage[unicode=true]{hyperref}

\hypersetup{
    colorlinks,
    citecolor=black,
    filecolor=black,
    linkcolor=black,
    urlcolor=black
}

\svnInfo $Id$

\begin{document}
\begin{titlepage}
\begin{flushright}
	{\huge
	\textbf{Техническое описание проекта по курсу ООАД}
\\[1cm]
	\textbf{KEDR}
\\[1cm]
	\textbf{Студенты ФИТ НГУ}

	\textbf{Кантеров Глеб Константинович}

	\textbf{Кузнецов Илья Владиславович}
\\[1cm]
	\textbf{группа 7201}
	}
\\[1cm]
	{\large \textbf{Версия 0.\svnInfoRevision}}
\end{flushright}
\end{titlepage}

\tableofcontents
\newpage

\section*{Техническое описание проекта по курсу ООАД}
\section{Введение}
\subsection{Цель}
Данный документ представляет собой техническое описание проекта KEDR и содержит основные требования к разрабатываемой в рамках проекта программной системе и описание архитектуры программного решения.

\subsection{Область действия}
Документ разработан в рамках проекта KEDR на основе стандартного шаблона Inteks SEP и предназначен для использования студентами ФИТ и преподавателями курса ООАД.

\subsection{Определения и сокращения}
\comment{[В этой таблице нужно перечислить все термины предметной области, используемые далее в документе. В тексте документа термины имеет смысл выделять курсивом. Текст, выделенный \replace{зеленым}, является ПРИМЕРОМ, в вашем проекте он может и должен быть другим.}

\comment{\textbf{Этот и прочие комментарии, выделенные синим, в финальной версии документа нужно удалить.}]}

\begin{table}[h]  %% FIXME, TODO: align caption to left, make colored headers (Термин, Описание)
\caption{Определения и сокращения}
\begin{tabular}{|p{4cm}|p{10cm}|} \hline
Термин & Описание \\ \hline
MusicBrainz & Открытая музыкальная энциклопедия (база данных) \\ \hline
MusicDNS    & Сервис звуковых отпечатков (база данных) \\ \hline
Звуковой отпечаток (acoustic fingerprint) & Цифровая характеристика, показывающая отличительные особенности звукового фрагмента \\ \hline
ПО & Программное обеспечение \\ \hline
\end{tabular}
\end{table}

\subsection{Ссылки}
%% FIXME
В тексте содержатся ссылки на следующие документы:

[1]	<Имя файла документа>, v<версия> --- <описание документа>

Ссылки приводятся в виде [N], где N --- номер документа в вышеприведенном списке.

\subsection{Краткое описание}
Содержание данного документа построено таким образом, чтобы дать ответ на следующие вопросы:

\begin{itemize}
	\item Какие проблемы предметной области должен решать будущий программный продукт
	\item Посредством какой функциональности системы будут достигнуто решение проблем предметной области
	\item Какова архитектура программного решения
\end{itemize}

Описание предметной области и проблем, для решения которых предназначен будущий программный продукт, приведены в разделе 2.
Раздел 3 содержит описание требований к программному решению, раздел 4 --- описание архитектуры выбранного решения.

\section{Предметная область проекта}
\comment{[Здесь должно быть дано краткое введение в предметную область проекта. Текст должен давать достаточно информации для того, чтобы непосвященный человек ознакомился с предметом, но не должен быть перегружен деталями]}

Проект KEDR направлен на упрощение содержания и структурирования цифровой музыкальной коллекции на компьютере.

\subsection{Существующие проблемы}
\comment{[Перечень объективных и субъективных проблем предметной области, побуждающих к выполнению задач данного проекта. Описание проблемы должно включать: 
	\begin{itemize}
		\item Суть проблемы;
		\item Порождающие ее причины и их влияние на участников (stakeholders)  предметной области; 
		\item Пути решения этой проблемы (через устранение соответствующих причин), которые достигаются в рамках данного проекта.]
	\end{itemize}}

\subsection{Предполагаемое решение}
\comment{[Здесь необходимо кратко описать, как именно предполагается решить проблемы предметной области.]}

\section{Требования к программному решению}
Данный раздел описывает требования к программной системе, разрабатываемой в рамках проекта KEDR.

\subsection{Роли}
Роль --- это что-то (например: другая система) или кто-то (например: человек) вне системы, которые взаимодействуют с ней. В предлагаемой к разработке системе идентифицированы следующие роли:
\replace{
	\begin{enumerate}
		\item <Роль1> --- <краткое описание роли>
		\item <Роль2> --- <краткое описание роли>
	\end{enumerate}
}

\subsection{Функциональные требования для роли Роль1}
\comment{[В этом пункте необходимо сделать описание требований к системе в соответствии с Use-Case моделью. Для каждой роли необходимо ввести отдельный пункт 2-го уровня, такой как 3.2]}

\subsubsection{<Use Case Name 1>}
\comment{[В этом пункте необходимо сделать описание данного Use-Case.]}

\subsubsection{<Use Case Name 2>}
\comment{[В этом пункте необходимо сделать описание данного Use-Case.]}

\subsection{Функциональные требования для роли Роль2}

\subsubsection{<Use Case Name 1>}
\comment{[В этом пункте необходимо сделать описание данного Use-Case.]}

\subsubsection{<Use Case Name 2>}
\comment{[В этом пункте необходимо сделать описание данного Use-Case.]}

\subsection{Нефункциональные требования}
\comment{[В этом пункте необходимо описать нефункциональные требования, такие как:
\begin{itemize}
\item Производительность
\item Масштабируемость
\item Ограничения по используемым компонентам
\item Необходимость миграции данных из legacy систем
\item И т.д.]
\end{itemize}}

\section{Обзор архитектуры}
Этот раздел описывает архитектуру системы.

\setcounter{subsection}{1}
\subsubsection{Компонентная модель системы}
\comment{[Здесь приводится Component diagram  - диаграмма компонентов системы, со связями между компонентами  и интерфейсами между ними, а также описание их взаимодействия. Для каждого компонента дается краткое описание его места и предназначения в системе]}

\paragraph{Компонент 1}
\comment{[Здесь приводится более подробное описание предназначения компонента и Package diagram – диаграмма пакетов, из которых состоит данный компонент. Обязательно выделение на диаграмме интерфейсов пакета, служащих для связи с другими пакетами (фасад пакета), а также ключевых классов, используемых другими пакетами в use-case реализациях]}


\paragraph{Компонент 2}
\comment{[Здесь приводится более подробное описание предназначения компонента и Package diagram – диаграмма пакетов, из которых состоит данный компонент. Обязательно выделение на диаграмме интерфейсов пакета, служащих для связи с другими пакетами (фасад пакета), а также ключевых классов, используемых другими пакетами в use-case реализациях]}

\subsubsection{Компоненты сторонних производителей}
\comment{[Здесь приводится список использованных компонент сторонних производителей, использованных при разработке системы, с указанием их предназначения в системе]}

\subsubsection{Схема развертывания приложения}
\comment{[Здесь приводится Deployment diagram  - диаграмма развертывания системы, со связями между узлами и указанием способа связи (протокола). На диаграмме обязательно указать, какие компоненты находятся на том или ином узле]}

\section{Допущения и ограничения}
\comment{[Краткое описание допущений,  которые подразумевает данный проект, и любых ограничений (например, по бюджету, участникам, требуемому оборудованию, срокам и т.п.), накладываемых на его выполнение.]}

\replace{Пример: При разработке проекта принято допущение, что число транзакций в единицу времени значительно (более чем  в 10 раз) снижается в ночное время, что позволяет в период с 01:00 до 6:00 производить автоматическое обновление программного обеспечения системы, требующее полной перезагрузки и остановки сервиса на период до 5 минут.}

\end{document}
